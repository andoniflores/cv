\documentclass[letterpaper,11pt]{article}

\usepackage{latexsym}
\usepackage[empty]{fullpage}
\usepackage{titlesec}
\usepackage{marvosym}
\usepackage[usenames,dvipsnames]{color}
\usepackage{verbatim}
\usepackage{enumitem}
\usepackage[hidelinks]{hyperref}
\usepackage{fancyhdr}
\usepackage[english]{babel}
\usepackage{tabularx}
\usepackage{fontawesome5}
\usepackage{multicol}
\setlength{\footskip}{5pt}
\setlength{\multicolsep}{-3.0pt}
\setlength{\columnsep}{-1pt}
\input{glyphtounicode}


%----------FONT OPTIONS----------
% sans-serif
% \usepackage[sfdefault]{FiraSans}
% \usepackage[sfdefault]{roboto}
% \usepackage[sfdefault]{noto-sans}
% \usepackage[default]{sourcesanspro}

% serif
% \usepackage{CormorantGaramond}
% \usepackage{charter}


\pagestyle{fancy}
\fancyhf{} % clear all header and footer fields
\fancyfoot{}
\renewcommand{\headrulewidth}{0pt}
\renewcommand{\footrulewidth}{0pt}

% Adjust margins
\addtolength{\oddsidemargin}{-0.6in}
\addtolength{\evensidemargin}{-0.5in}
\addtolength{\textwidth}{1.19in}
\addtolength{\topmargin}{-.7in}
\addtolength{\textheight}{1.4in}

\urlstyle{same}

\raggedbottom
\raggedright
\setlength{\tabcolsep}{0in}

% Sections formatting
\titleformat{\section}{
  \vspace{-4pt}\scshape\raggedright\large\bfseries
}{}{0em}{}[\color{black}\titlerule \vspace{-5pt}]

% Ensure that generate pdf is machine readable/ATS parsable
\pdfgentounicode=1

%-------------------------
% Custom commands
\newcommand{\resumeItem}[1]{
  \item\small{
    {#1 \vspace{-2pt}}
  }
}

\newcommand{\classesList}[4]{
    \item\small{
        {#1 #2 #3 #4 \vspace{-2pt}}
  }
}

\newcommand{\resumeSubheading}[4]{
  \vspace{-2pt}\item
    \begin{tabular*}{1.0\textwidth}[t]{l@{\extracolsep{\fill}}r}
      \textbf{#1} & \textbf{\small #2} \\
      \textit{\small#3} & \textit{\small #4} \\
    \end{tabular*}\vspace{-7pt}
}

\newcommand{\resumeSubSubheading}[2]{
    \item
    \begin{tabular*}{0.97\textwidth}{l@{\extracolsep{\fill}}r}
      \textit{\small#1} & \textit{\small #2} \\
    \end{tabular*}\vspace{-7pt}
}

\newcommand{\resumeProjectHeading}[2]{
    \item
    \begin{tabular*}{1.001\textwidth}{l@{\extracolsep{\fill}}r}
      \small#1 & \textbf{\small #2}\\
    \end{tabular*}\vspace{-7pt}
}

\newcommand{\resumeSubItem}[1]{\resumeItem{#1}\vspace{-4pt}}

\renewcommand\labelitemi{$\vcenter{\hbox{\tiny$\bullet$}}$}
\renewcommand\labelitemii{$\vcenter{\hbox{\tiny$\bullet$}}$}

\newcommand{\resumeSubHeadingListStart}{\begin{itemize}[leftmargin=0.0in, label={}]}
\newcommand{\resumeSubHeadingListEnd}{\end{itemize}}
\newcommand{\resumeItemListStart}{\begin{itemize}}
\newcommand{\resumeItemListEnd}{\end{itemize}\vspace{-5pt}}

%-------------------------------------------
%%%%%%  RESUME STARTS HERE  %%%%%%%%%%%%%%%%%%%%%%%%%%%%


\begin{document}

%----------HEADING----------
% \begin{tabular*}{\textwidth}{l@{\extracolsep{\fill}}r}
%   \textbf{\href{http://sourabhbajaj.com/}{\Large Sourabh Bajaj}} & Email : \href{mailto:sourabh@sourabhbajaj.com}{sourabh@sourabhbajaj.com}\\
%   \href{http://sourabhbajaj.com/}{http://www.sourabhbajaj.com} & Mobile : +1-123-456-7890 \\
% \end{tabular*}

\begin{center}
    {\Huge \scshape Andoni Flores} \\ \vspace{1pt}
    Santiago, Chile  \\ \vspace{1pt}
    \small \raisebox{-0.1\height}\faPhone\ 991834570 ~ \href{mailto:andonibalsebre@gmail.com}{\raisebox{-0.2\height}\faEnvelope\  \underline{andonibalsebre@gmail.com}} ~ 
    \href{https://linkedin.com/in/andonifloresbalsebre/}{\raisebox{-0.2\height}\faLinkedin\ \underline{linkedin.com/in/andonifloresbalsebre/}}  ~
    \href{https://github.com/andoniflores}{\raisebox{-0.2\height}\faGithub\ \underline{github.com/andoniflores}}
    \vspace{-8pt}
\end{center}
%

%-----------ABOUT-----------
\section{Sobre mi}
  \resumeSubHeadingListStart
    \small{\item{Experimentado Ingeniero de software con tres años y medio de experiencia de trabajo
        desarrollando aplicaciones web como desarrollador full-stack utilizando Python, Flask,
        ReactJS y AWS. Habilidad comprobada trabajando eficientemente en un equipo y entregando
        proyectos a tiempo. Fuertes habilidades de comunicación demostrada a través de colaboración
        con product managers y stakeholders. Dedicado a desarrollar software de alta calidad y
        mejorar la experiencia de usuario. Recientemente he realizado una transición a la industria 
        de videojuegos, donde he perfeccionado mis habilidades de ingeniero backend en un entorno exigente
        y dinámico, demostrando también adaptabilidad y una rápida curva de aprendizaje en un entorno de ritmo
        acelerado.}}
  \resumeSubHeadingListEnd

%-----------PROGRAMMING SKILLS-----------
\section{Habilidades Técnicas}
 \begin{itemize}[leftmargin=0.15in, label={}]
    \small{\item{
     \textbf{Lenguajes}{: Python, C\#, HTML/CSS, JavaScript, SQL, Java, NoSQL, Typescript} \\
     \textbf{Herramientas de Desarrollo}{: AWS, Postman, Git(GitHub), MySQL, Redis, Jenkins} \\
     \textbf{Tecnologías/Frameworks}{: API REST, Flask, ReactJS, Tailwind, .NET} \\
    }}
 \end{itemize}
 \vspace{-16pt}


%-----------EXPERIENCE-----------
\section{Experiencia laboral}
  \resumeSubHeadingListStart
    \resumeSubheading
      {Gala Games (The Walking Dead: Empires)}{Agosto 2024 - Diciembre 2024}
      {Backend Engineer}{Santiago, Chile}
      \resumeItemListStart
        \resumeItem{Corrección de bugs altamente esperado por la comunidad, que requería conocimiento sobre el cliente de Unity y 
        el servidor de backend utilizando C\# y .NET. Contribuyendo en la mejora del pulido del juego y disminuyendo la frustración en la jugabilidad.}
        \resumeItem{Entrega de builds de staging del proyecto utilizando Jenkins, estas entregas serían usadas por el equipo para 
        realizar rondas de testeo.}
      \resumeItemListEnd
    \resumeSubheading
      {ServiceRocket}{Abril 2022 -- Julio 2023}
      {Software Engineer}{Santiago, Chile}
      \resumeItemListStart
        \resumeItem{Desarrollo y mantención de aplicaciones web principalmente para el marketplace de Atlassian, utilizando Python, Flask, ReactJS, AWS and CSS}
        \resumeItem{Liderazgo en el desarrollo e implementación de recordatorios para responder
        encuestas para la aplicación Surveys for JSM, utilizando AWS EventBridge para automatizar
        la ejecución de funciones AWS Lambda que procesan los recordatorios pendientes y se logró
        obtener una tasa del 30\% de adopción posicionando esta funcionalidad en el top 4 de
        funcionalidades para Surveys for JSM.}
        \resumeItem{Colaboración con product managers y stakeholders para recolectar requerimientos y asegurar la implementación exitosa de nuevas funcionalidades.}
        \resumeItem{Mantención y mejoras de las funcionalidades existentes en dos aplicaciones, asegurando un alto desempeño y satisfacción de usuarios.}
        \resumeItem{Participación en code reviews, entregando retroalimentación para mejorar la calidad del codigo y mantener buenas practicas.}
        \resumeItem{Contribución activa al proceso de desarrollo ágil, incluyendo sprint planning, stand-ups y retrospectivas.}
      \resumeItemListEnd
    \resumeSubheading
      {Shift}{Febrero 2020 -- Abril 2022}
      {Software Engineer}{Santiago, Chile}
      \resumeItemListStart
        \resumeItem{Desarrollo y mantenimiento del producto principal de Shift, una aplicación web de administración de recursos humanos.}
        \resumeItem{Liderazgo en el desarrollo de una nueva funcionalidad para el modulo de cambio de turnos utilizando Javascript, permitiendo a los administradores manejar dias con doble turno.}
        \resumeItem{Se estableció una funcionalidad crítica que transformaba data en la aplicación
        a un archivo de hoja de cálculo, acortando el tiempo destinado a esta operación en 
        aproximadamente un 50\%}
        \resumeItem{Modernización de la interfaz de usuario utilizando React.}
      \resumeItemListEnd  
  \resumeSubHeadingListEnd
\vspace{-16pt}


%-----------EDUCATION-----------
\section{Educación}
  \resumeSubHeadingListStart
    \resumeSubheading
      {Universidad de Tarapacá}{Mar. 2013 -- Dic 2019}
      {Ingeniería Civil en Computación e Informática}{Arica, Chile}
    \resumeSubheading
      {Universitat de Valencia}{Sep. 2018 -- Ene 2019}
      {Programa de Intercambio}{Valencia, España}
  \resumeSubHeadingListEnd


%-----------PROJECTS-----------
%\section{Projects}
%    \vspace{-5pt}
%    \resumeSubHeadingListStart
%      \resumeProjectHeading
%          {\textbf{Gym Reservation Bot} $|$ \emph{Python, Selenium, Google Cloud Console}}{January 2021}
%          \resumeItemListStart
%            \resumeItem{Developed an automatic bot using Python and Google Cloud Console to register myself for a timeslot at my school gym.}
%            \resumeItem{Implemented Selenium to create an instance of Chrome in order to interact with the correct elements of the web page.}
%            \resumeItem{Created a Linux virtual machine to run on Google Cloud so that the program is able to run everyday from the cloud.}
%            \resumeItem{Used Cron to schedule the program to execute automatically at 11 AM every morning so a reservation is made for me.}
%          \resumeItemListEnd
%          \vspace{-13pt}
%      \resumeProjectHeading
%          {\textbf{Ticket Price Calculator App} $|$ \emph{Java, Android Studio}}{November 2020}
%          \resumeItemListStart
%            \resumeItem{Created an Android application using Java and Android Studio to calculate ticket prices for trips to museums in NYC.}
%            \resumeItem{Processed user inputted information in the back-end of the app to return a subtotal price based on the tickets selected.}
%            \resumeItem{Utilized the layout editor to create a UI for the application in order to allow different scenes to interact with each other.}
%          \resumeItemListEnd 
%          \vspace{-13pt}
%          \resumeProjectHeading
%          {\textbf{Transaction Management GUI} $|$ \emph{Java, Eclipse, JavaFX}}{October 2020}
%          \resumeItemListStart
%            \resumeItem{Designed a sample banking transaction system using Java to simulate the common functions of using a bank account.}
%            \resumeItem{Used JavaFX to create a GUI that supports actions such as creating an account, deposit, withdraw, list all acounts, etc.}
%            \resumeItem{Implemented object-oriented programming practices such as inheritance to create different account types and databases.}
%          \resumeItemListEnd 
%    \resumeSubHeadingListEnd
%\vspace{-15pt}
\end{document}
